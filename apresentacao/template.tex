\documentclass[xcolor=dvipsnames,10pt]{beamer}
\usepackage[utf8]{inputenc}
\usepackage[T1]{fontenc}
\usepackage[brazilian]{babel}
\usepackage{etoolbox}
\usepackage{graphicx}
\usepackage{ragged2e}
\usepackage{tikz}

\makeatletter
\patchcmd{\beamer@sectionintoc}{\vskip1.5em}{\vskip0.5em}{}{}
\makeatother

\usefonttheme[professional]{serif}

\justifying

%----------------------------------------------------------------------------------------------------------------------------------

% JUSTIFICAR TEXTO DAS LISTAS DE ITENS

\makeatletter
\renewcommand{\itemize}[1][]{%
  \beamer@ifempty{#1}{}{\def\beamer@defaultospec{#1}}%
  \ifnum \@itemdepth >2\relax\@toodeep\else
    \advance\@itemdepth\@ne
    \beamer@computepref\@itemdepth% sets \beameritemnestingprefix
    \usebeamerfont{itemize/enumerate \beameritemnestingprefix body}%
    \usebeamercolor[fg]{itemize/enumerate \beameritemnestingprefix body}%
    \usebeamertemplate{itemize/enumerate \beameritemnestingprefix body begin}%
    \list
      {\usebeamertemplate{itemize \beameritemnestingprefix item}}
      {\def\makelabel##1{%
          {%
            \hss\llap{{%
                \usebeamerfont*{itemize \beameritemnestingprefix item}%
                \usebeamercolor[fg]{itemize \beameritemnestingprefix item}##1}}%
          }%
        }%
      }
  \fi%
  \beamer@cramped%
  \justifying% NEW
  %\raggedright% ORIGINAL
  \beamer@firstlineitemizeunskip%
}
\makeatother

%%%%%%%%%%%%%%%%%%%%%%%%%%%%
%%TAMANHO DE BLOCOS VARIÁVEIS
\newenvironment<>{varblock}[2][1.2\textwidth]{%
  \setlength{\textwidth}{#1}
  \begin{actionenv}#3%
    \def\insertblocktitle{#2}%
    \par%
    \usebeamertemplate{block begin}}
  {\par%
    \usebeamertemplate{block end}%
  \end{actionenv}}
%%-------------------------
%%%%%%%%%%%%%%%%%%%%%%%%%%%%%%%%%%%%%%%%%%%%%%%%%%%%%%%%%%%%%%%%%%%%%%%%%%%%%%%%
%COR DO TEMPLATE SEMELHANTE AO LOGO
%%%%%%%%%%%%%%%%%%%%%%%%%%%%%%%%%%%%%%%%%%%%%%%%%%%%%%%%%%%%%%%%%%%%%%%%%%%%%%%%
\definecolor{azul}{rgb}{0.228,0.632,0.816}

%%%%%%%%%%%%%%%%%%%%%%%%%%%%%%%%%%%%%%%%%%%%%%%%%%%%%%%%%%%%%%%%%%%%%%%%%%%%%%%%
%CORES E ESTRUTURAS DE ELEMENTOS DIVERSOS
%%%%%%%%%%%%%%%%%%%%%%%%%%%%%%%%%%%%%%%%%%%%%%%%%%%%%%%%%%%%%%%%%%%%%%%%%%%%%%%%
\setbeamercolor{frametitle}{fg=black}
\setbeamercolor{headline}{fg=black}
\setbeamercolor{footline}{fg=black}
\setbeamercolor{block body}{use=structure,bg=white}
\setbeamercolor{block title}{use=structure,bg=azul,fg=darkgray}
\setbeamercolor{item}{fg=azul}
\setbeamercolor{alerted text}{fg=azul}
\setbeamercolor{section in toc}{fg=black}
\setbeamercolor{title}{fg=black}
\setbeamercolor{subtitle}{fg=black}
\setbeamertemplate{blocks}[rounded][shadow=true]

%%%%%%%%%%%%%%%%%%%%%%%%%%%%%%%%%%%%%%%%%%%%%%%%%%%%%%%%%%%%%%%%%%%%%%%%%%%%%%%%
%PROPRIEDADES VISUAIS DE ELEMENTOS DIVERSOS
%%%%%%%%%%%%%%%%%%%%%%%%%%%%%%%%%%%%%%%%%%%%%%%%%%%%%%%%%%%%%%%%%%%%%%%%%%%%%%%%
\useoutertheme[]{default}
\addtobeamertemplate{block begin}{}{\justifying} % Justify all blocks

\setbeamertemplate{section in toc}[circle]
\setbeamertemplate{subsection in toc}[square]
\setbeamertemplate{itemize item}{$\bullet$}
\setbeamertemplate{navigation symbols}{}
\setbeamercovered{invisible}
\setbeamertemplate{sections in toc}[ball]

%%%%%%%%%%%%%%%%%%%%%%%%%%%%%%%%%%%%%%%%%%%%%%%%%%%%%%%%%%%%%%%%%%%%%%%%%%%%%%%%
%LOGOS POSICIONADOS NO RODAPÉ
%%%%%%%%%%%%%%%%%%%%%%%%%%%%%%%%%%%%%%%%%%%%%%%%%%%%%%%%%%%%%%%%%%%%%%%%%%%%%%%%
%\textsc{\setbeamertemplate{footline}{
%  \vspace{0.5em}
%  \tikz \fill [azul] (0,0) rectangle (15, 2pt);
%
%\vspace{-1.3em}
%\begin{center}
%\includegraphics[height=0.5cm]{logo-iff.png}\hspace{0.4em}
%\includegraphics[height=0.5cm]{logo-iprj.png}\hspace{0.4em}
%\includegraphics[height=0.5cm]{logo-uerj.png}\hspace{0.4em}
%\includegraphics[height=0.5cm]{logo-uesc.png}\hspace{0.4em}
%\includegraphics[height=0.5cm]{logo-capes.png}\hspace{0.4em}
%\includegraphics[height=0.5cm]{logo-cnpq.png}\hspace{0.4em}
%\includegraphics[height=0.5cm]{logo-faperj.png}\hspace{0.4em}
%\includegraphics[height=0.5cm]{logo-buzios.png}\vspace{-0.5em}
%\end{center}
%}}

%%%%%%%%%%%%%%%%%%%%%%%%%%%%%%%%%%%%%%%%%%%%%%%%%%%%%%%%%%%%%%%%%%%%%%%%%%%%%%%%
%LOGO DO EVENTO COMO TÍTULO DE TODOS OS SLIDES
%%%%%%%%%%%%%%%%%%%%%%%%%%%%%%%%%%%%%%%%%%%%%%%%%%%%%%%%%%%%%%%%%%%%%%%%%%%%%%%%
\setbeamertemplate{headline}{

\vspace{-1em}
\begin{center}
\includegraphics[width=0.8\textwidth]{logomcsul.jpg}
\end{center}
}
%%%%%%%%%%%%%%%%%%%%%%%%%%%%%%%%%%%%%%%%%%%%%%%%%%%%%%%%%%%%%%%%%%%%%%%%%%%%%%%%
% INICIO DOS SLIDES
%%%%%%%%%%%%%%%%%%%%%%%%%%%%%%%%%%%%%%%%%%%%%%%%%%%%%%%%%%%%%%%%%%%%%%%%%%%%%%%%
\title{\textbf{\large AVALIAÇÃO DE DESEMPENHO DO ALGORITMO DE EVOLUÇÃO DIFERENCIAL ASSOCIADO AO DESIGN CONSTRUTAL PARA A OTIMIZAÇÃO DE UMA CAVIDADE EM FORMA DE DUPLO-T
}}

\author{\small G. V. Gonzales,  L. A. Isoldi, L. A. O. Rocha, E. D. dos Santos e \\
A. J. Silva Neto}
\institute{\vspace{-0.3cm}\scriptsize Programa de Pós-Graduação em Modelagem Computacional - FURG}
%%\subtitle{\vspace{0.3cm}\small Métodos Computacionais\vspace{-0.3cm}}
\date{\vspace{-0.4cm}\scriptsize Outubro de 2018}
\begin{document}

\frame{\titlepage}
\frame{\tableofcontents}

\section{Introdução}
\subsection{Motivação}
\begin{frame}{}
	\begin{block}{\textbf{\textsc{MOTIVAÇÃO}}}
	\justifying
	\setlength{\parindent}{1cm}
	Com a miniaturização dos circuitos eletrônicos e desenvolvimento de dispositivos cada vez mais compactos, técnicas tradicionais de troca térmica por convecção forçada não são mais suportadas. Alternativas apontam para cavidades ou caminhos com material de alta condutibilidade.
	\end{block}
\end{frame}
\subsection{Objetivos}
\begin{frame}{}
	\begin{block}{\textbf{\textsc{OBJETIVOS}}}
	\justifying
	\setlength{\parindent}{1cm}
		\begin{itemize}
		\item Otimizar parcialmente uma cavidade em forma de Duplo-T;
		\item Comparar os resultados de diferentes versões do algoritmo de Evolução Diferencial (ED)
		\item Analisar diferentes parâmetros do algoritmo ED;
		\item Avaliar estatisticamente os resultados da reprodução dos efeitos dos graus de liberdade sobre a geometria ótima e a temperatura máxima mínima;
		\item Recomendar a configuração de parâmetros mais adequada ao problema de otimização;
	\end{itemize}
	\end{block}
\end{frame}

\subsection{Breve Estado da Arte}
	\begin{frame}{}
	\begin{block}{\textbf{\textsc{BREVE ESTADO DA ARTE}}}	
\justifying
	\setlength{\parindent}{1cm}
		\begin{itemize}
		\item Cavidade em formato de "C" e "T"  em Biserni et. al. (2004). 
		\item Cavidade em forma de "H"  em Biserni et. al.(2007).
		\item Cavidade em forma de "Y"  em (Lorenzini et. al.(2011).
		\item Cavidade em forma de "Y" aplicação do Algoritmo Genético  em Lorenzini et. al.(2014).
		\item Comparação entre aplicação do SA com GA na otimização da cavidade em forma de Y  em Gonzales et. al.(2015a).
		\item Otimização parcial até 3 graus de liberdade da cavidade em duplo-T  em Gonzales et. al. (2015b).
	\end{itemize}
	\end{block}
\end{frame}

\section{Modelagem Matemática e Numérica}

\begin{frame}[plain]
\begin{block}{\textbf{MODELAGEM MATEMÁTICA E NUMÉRICA}}
\begin{figure}[h!]
			\centering
			\includegraphics[width=0.9\linewidth]{../imgs/duplo_t.png}
			\caption{ {\small Domínio Computacional da Cavidade em Forma de Duplo-T.}}
			\label{figure01}
		\end{figure}		
\end{block}
\end{frame}

%\begin{frame}{}
%\begin{block}{TESTE}
%	\begin{figure}[hbtp]
%		\caption{}
%		\centering
%		\includegraphics[scale=0.5]{../../imgs/agrade.png}
%		\end{figure}
%
%teste
%\end{block}
%\end{frame}


\begin{frame}{}
\begin{block}{\textbf{\textsc{MODELAGEM MATEMÁTICA E NUMÉRICA}}}
\justifying
	\setlength{\parindent}{1cm}
		\textbf{Hipóteses Simplificativas:}
		\small
		\begin{enumerate}
			\item Regime Permanente
			\item Geração uniforme de calor
			\item Condutividade térmica constante
			\item Domínio bidimensional
		\end{enumerate}
		\normalsize
		\begin{equation}
		\frac{\partial}{\partial x}\left( k\frac{\partial \theta}{\partial x}\right)+
		\frac{\partial}{\partial y}\left(k\frac{\partial \theta}{\partial y}\right)+
		\frac{\partial}{\partial z}\left(k\frac{\partial \theta}{\partial z}\right)+ 
		q^{'''}=\rho C_p\frac{\partial \theta}{\partial t}\label{calor}
		\end{equation}
		\begin{equation}
		\frac{\partial ^{2} \theta}{\partial x^{2}}+\frac{\partial ^{2} \theta}{\partial y^{2}}+\frac{q^{'''}}{k}=0\label{calor}
		\end{equation}
	\end{block} 
\end{frame}

\begin{frame}{}
	\begin{block}{\textbf{\textsc{MODELAGEM MATEMÁTICA E NUMÉRICA}}}
	\textbf{Restrições: }
		\begin{equation}
			A = HL \label{area_total}
		\end{equation}
		\begin{equation}
			A_{c} = A_{0} + 2A_{1} + 2A_{2} \label{area_cavidade}
		\end{equation}\begin{equation}
			\phi_{c} = A_{c}/A \label{fi}
		\end{equation}
		
	\end{block} 
\end{frame}


\begin{frame}
\begin{varblock}[11.5cm]{\textbf{\textsc{MODELAGEM MATEMÁTICA E NUMÉRICA}}}\justifying
	\setlength{\parindent}{1cm}
\textbf{Problema na forma adimensional:}
		\begin{equation}
				\tilde{x},\tilde{y},\tilde{H}_{0},\tilde{H}_{1},\tilde{H}_{2},\tilde{L}_{0},\tilde{L}_{1},\tilde{L}_{2},\tilde{H},\tilde{L},\tilde{S}_{1} = \frac{x,y,H_{0},H_{1},H_{2},L_{0},L_{1},L_{2},H,L,S_{1}}{A^{1/2}}\label{vadim} 
		\end{equation}
		\begin{equation}
				\frac{\partial ^{2}\tilde{\theta}}{\partial \tilde{x}^{2}}+\frac{\partial ^{2} \tilde{\theta}}{\partial \tilde{y}^{2}}+1=0\label{calor}
		\end{equation}
		\begin{equation}
			\tilde{\theta}_{max}=\frac{\theta_{max}-\theta_{min}}{q^{'''}\cdot\frac{A}{k}}\label{fo}
		\end{equation}
		\end{varblock}
\end{frame}
\begin{frame}
\begin{block}{\textbf{\textsc{MODELAGEM MATEMÁTICA E NUMÉRICA}}}
\justifying
	\setlength{\parindent}{1cm}
		A função representada pela Eq. \ref{fo} é resolvida numericamente através da resolução da Eq. \ref{calor} para a determinação dos os campos de temperatura em todo o domínio computacional para diferentes configurações de ($H$, $L$, $H_{0}$, $L_{0}$, $H_{1}$, $L_{1}$, $H_{2}$, $L_{2}$ e $S_{1}$) e calculando o $\tilde{\theta}_{max}$ para minimizar o seu valor através da variação da configuração geométrica.\\ A solução numérica é dada pela aplicação do método de Elementos Finitos (FEM), baseado em elementos triangulares, desenvolvido no ambiente MATLAB$^{\tiny ®}$, com o pacote PDE (partial-differential-equations) toolbox.\\ A malha utilizada é não-uniforme em ambos eixos $x$ e $y$, e varia de uma geometria para outra. O tamanho é de  80649 mil elementos.
	\end{block}
\end{frame}
\section{Otimização}
\begin{frame}
	\begin{block}{\textbf{\textsc{OTIMIZAÇÃO}}}
	\justifying
	\setlength{\parindent}{1cm}
	A metodologia de otimização aplicada neste trabalho utiliza-se do método Constructal Design associado as meta-heurísticas Differential Evolution (DE) e Simulated Annealing (SA). 
	\begin{itemize}
		\item \textbf{Constructal Design}: para definição dos objetivos, restrições, Graus de Liberdade (GL) e espaço de busca.
		\item \textbf{Algoritmos de Otimização}: neste trabalho são executadas diferentes versões do   algoritmo ED com variações nos parâmetros do algoritmo.
		\item \textbf{Avaliação dos Resultados}: São comparados os valores de média e desvio padrão encontrados entre 30 execuções de cada algoritmo.
	\end{itemize}
	\end{block}
\end{frame}
\subsection{Design Construtal}
\begin{frame}{}
	\begin{block}{\textbf{CONSTRUCTAL DESIGN}}
		\justifying
		\setlength{\parindent}{1cm}
		\textbf{Definição dos Graus de Liberdade e Restrições:}
		\begin{itemize}
			\item Nove variáveis ($H$, $L$, $H_{0}$, $L_{0}$, $H_{1}$, $L_{1}$, $H_{2}$, $L_{2}$ e $S_{1}$);
			\item Quatro restrições ($A$, $A_c$, $A_1$ e $A_2$);
			\begin{equation}
				\phi_c = A_c/A= \tilde{H_{0}}\tilde{L_{0}}+2\phi_1+2\phi_2
			\end{equation}
			\begin{equation}
				\phi_1 = \tilde{H_{1}}\tilde{L_{1}} 
			\end{equation}
			\begin{equation}
				\phi_2 = \tilde{H_{2}}\tilde{L_{2}}
			\end{equation}
			\item Temos cinco Graus de liberdade ($H/L$, $H_{0}/L_{0}$, $H_{1}/L_{1}$, $H_{2}/L_{2}$ e $S_{1}/H_{0}$) para o fechamento das equações;
			
		\end{itemize}
	\end{block}
\end{frame}
\begin{frame}{}
	\begin{block}{\textbf{CONSTRUCTAL DESIGN}}
		\justifying
		\setlength{\parindent}{1cm}
		\begin{itemize}
\item Durante o processo de otimização, foram mantidos constantes os valores das restrições ($\phi_c = 0.1,  \phi_1 = \phi_2 = 0.015$) 
\item Para a otimização de 4 GLs, o grau de liberdade $H/L$ foi variado entre $0.3 =< H/L<= 30$;
			\item Sendo otimizados os graus de liberdade: $H_{0}/L_{0}$, $H_{1}/L_{1}$, $H_{2}/L_{2}$ e $S_{1}/H_{0}$;
		\end{itemize}
	\end{block}
\end{frame}

\subsection{Configuração dos Algoritmos}
\begin{frame}{}
\begin{block}{
\textbf{\textsc{CONFIGURAÇÃO DOS ALGORITMOS}}}
			\begin{table}[H] % !htbp
			\caption{Configurações das Diferentes Versões do Algoritmo ED }
			\label{table1}
			\vspace{5pt}
			\centering{}
			\begin{tabular*}{\textwidth}{@{\extracolsep{\fill}}ccccc}        % {0.8\textwidth}
			\hline
				Parâmetros & ED1 & ED2 & ED3 & ED4 \tabularnewline
			\hline
				Amplificação $F$ & 1.5 & 2.0 & 1.5 & 2.0\tabularnewline
			%\hline 
				Cruzamento & 0.7 & 0.9 & 0.7 & 0.9\tabularnewline
			%\hline
				Mutação & rand/1/bin & rand/1/bin & best/2/bin & best/2/bin \tabularnewline
			\hline
			\end{tabular*}
			\end{table}
			\end{block}
\end{frame}	
\begin{frame}{}
\begin{block}{
\textbf{\textsc{CONFIGURAÇÃO DOS ALGORITMOS}}}

\begin{table}[htbp]
\small
\centering
\caption{\small Combinações dos Parâmetros de Tamanho da População $(NP)$ e Número de Gerações $(G)$}
\begin{tabular}{ccc}
\hline
$(NP)$ & $(G)$ & Total de Iterações \\
\hline
$5$ & $10$ & $50$	\\
$10$ & $10$ & $100$	\\
$10$ & $15$ & $150$	\\
$15$ & $20$ & $300$	\\
\hline
\end{tabular}
  \label{tab:popgeracao}
\end{table}

\end{block}
\end{frame}	
\section{Resultados}


\begin{frame}[plain]
\noindent
\begin{varblock}[11.5cm]{\textbf{\textsc{PRINCIPAIS RESULTADOS}}}
\begin{figure}[htbp]
\hspace{-1cm}
\quad
\includegraphics[scale=.75]{../imgs/plot_de1_rdata_std.png}
\quad
\includegraphics[scale=.75]{../imgs/plot_de2_rdata_std.png}
\center
\caption{\fontsize{10pt}{\baselineskip}\selectfont\centering{ Média e Desvio Padrão do Efeito de $H/L$ sobre $({\tilde{\theta}}_{max})_{4m}$ registrados por diferentes versões do algoritmo ED com diferentes combinações de população e gerações: a)ED1 b)ED2}}
\label{destdinner}
\end{figure}
\end{varblock}
\end{frame}

\begin{frame}[plain]
\noindent
\begin{varblock}[11.5cm]{\textbf{\textsc{PRINCIPAIS RESULTADOS}}}
\begin{figure}[htbp]
\hspace{-1cm}
\quad
\includegraphics[scale=.75]{../imgs/plot_de3_rdata_std.png}
\quad
\includegraphics[scale=.75]{../imgs/plot_de4_rdata_std.png}
\center
\caption{\fontsize{10pt}{\baselineskip}\selectfont\centering{ Média e Desvio Padrão do Efeito de $H/L$ sobre $({\tilde{\theta}}_{max})_{4m}$ registrados por diferentes versões do algoritmo ED com diferentes combinações de população e gerações: c)ED3 d)ED4}}
\label{destdinner}
\end{figure}
\end{varblock}
\end{frame}


\begin{frame}[plain]
\noindent
\begin{varblock}[11.5cm]{\textbf{\textsc{PRINCIPAIS RESULTADOS}}}
\begin{figure}[htbp]
\hspace{-1cm}
\quad
\includegraphics[scale=.75]{../imgs/plot_deall50_rdata_std_003-05.png}
\quad
\includegraphics[scale=.75]{../imgs/plot_deall100_rdata_std_003-05.png}
\center
\caption{\fontsize{10pt}{\baselineskip}\selectfont\centering{Média e Desvio Padrão do Efeito de $H/L$ sobre $({\tilde{\theta}}_{max})_{4m}$ para $H/L\leqslant0.5$, registrados por diferentes versões do algoritmo ED com diferentes combinações de população e gerações: a)$NP=5$, $G=10$ b)$NP=10$, $G=10$.}}
\label{destdinner}
\end{figure}
\end{varblock}
\end{frame}

\begin{frame}[plain]
\noindent
\begin{varblock}[11.5cm]{\textbf{\textsc{PRINCIPAIS RESULTADOS}}}
\begin{figure}[htbp]
\hspace{-1cm}
\quad
\includegraphics[scale=.75]{../imgs/plot_deall150_rdata_std_003-05.png}
\quad
\includegraphics[scale=.75]{../imgs/plot_deall300_rdata_std_003-05.png}
\center
\caption{\fontsize{10pt}{\baselineskip}\selectfont\centering{Média e Desvio Padrão do Efeito de $H/L$ sobre $({\tilde{\theta}}_{max})_{4m}$ para $H/L\leqslant0.5$, registrados por diferentes versões do algoritmo ED com diferentes combinações de população e gerações: c) $NP=10$, $G=15$ d) $NP=15$, $G=20$.}}
\label{destdinner}
\end{figure}
\end{varblock}
\end{frame}



\begin{frame}[plain]
\noindent
\begin{varblock}[11.5cm]{\textbf{\textsc{PRINCIPAIS RESULTADOS}}}
\begin{figure}[htbp]
\hspace{-1cm}
\quad
\includegraphics[scale=.75]{../imgs/plot_deall150_rdata_std_h2l2.png}
\quad
\includegraphics[scale=.75]{../imgs/plot_deall150_rdata_std_h1l1.png}
\center
\caption{\fontsize{10pt}{\baselineskip}\selectfont\centering{Média e Desvio Padrão do Efeito de $H/L$ sobre a geometria ótima, registrados por diferentes versões do algoritmo ED com limite de 150 iterações:  b) Efeito de $H/L$ sobre ${(H_{2}/L_{2})_{3o}}$, c) Efeito de $H/L$ sobre ${(H_{1}/L_{1})_{2o}}$.}}
\label{destdinner}
\end{figure}
\end{varblock}
\end{frame}

\section{Conclusão}

\begin{frame}
\begin{block}{\textbf{\textsc{CONCLUSÃO}}}
\justifying
		\setlength{\parindent}{1cm}
		\begin{itemize}
			\item Dentre os algoritmos pesquisados e as configurações de parâmetros avaliadas, as versões ED2 e ED3 apresentaram um desempenho inferior as versões do ED1 e ED4.
			\item Melhores resultados e menor desvio padrão para os algoritmos com  os parâmetros de cruzamento $(CR)$ de 0.9 e fator de amplificação $(F)$ de 1.5;
			\item Portanto, para o problema de interesse, esses são os parâmetros recomendados para o algoritmo ED, pois foram aqueles que reproduziram de maneira mais precisa as curvas de efeito dos graus de liberdade sobre a geometria ótima e performance térmica do problema.
		\end{itemize}
\end{block}
\end{frame}

\section{Referências}

\begin{frame}

\begin{block}{\textbf{\textsc{REFERÊNCIAS}}}
\begin{thebibliography}{99}
\fontsize{8}{0}\selectfont

\bibitem[Bejan, 1996]{Bejan}
A. Bejan, Constructal-theory Network of Conducting Path for Cooling a Heat Generating Volume. {\it Int. J. Heat Mass Transfer}, vol. 40, n. 4, pp.799-816, 1996.

\bibitem[Biserni et al., 20104]{Biserni1}
C. Biserni, L. A. O. Rocha, A. Bejan, Inverted Fins: Geometric Optimization of the Intrusion Into a Conducting Wall. {\it Int. J. Heat and Mass Transfer}, 47, pp. 2577-2586, 2004.

\bibitem[Gonzales et al., 2015]{Gonzales2}
G. V. Gonzales, E. D. Dos Santos, L. A. Isoldi, E. da S. D. Estrada,L. A. O. Rocha, Constructal Design of Isothermal Double-T Shaped Cavity By Means of Simulated Annealing. In {\it Proceedings of the XXXVI Iberian Latin-American Congress on Computational Methods in Engineering}, Rio de Janeiro, RJ, Brazil, 2015.

\bibitem[Kirkpatrick et al.,1983]{Kirk}
S. Kirkpatrick, C. D. Gelatt, M. P. Vecchi, M. P., Optimization by Simulated Annealing. {\it Science}, New Series., v. 220, No 4598, pp 671-680, 1983.
\end{thebibliography}
\end{block}
\end{frame}

\begin{frame}
\begin{block}{\textbf{\textsc{REFERÊNCIAS}}}
\begin{thebibliography}{99}
\fontsize{8}{0}\selectfont

\bibitem[Lorenzini et al., 2014]{Lorenzini4}
G. Lorenzini, C. Biserni, E. da S. D. Estrada, E. D. Dos Santos, L. A. Isoldi, L. A. O. Rocha, Genetic Algorithm Applied to Geometric Optimization of Isothermal Y-Shaped Cavities. {\it Journal of Electronic Packaging }, vol 136, p. 031011-031011-9, 2014.

\bibitem[Metropolis et al.,1953]{Metropolis}
N. Metropolis, A. W. Rosenbluth, M. N. Rosenbluth, A. H. Teller, E. Teller, Equation of State Calculations by Fast Computing Machines. {\it The Journal of Chemical Physics.}, v 21, p 1088-1092, 1953.

\bibitem[Storn and Price., 1997]{Storn}
R. Storn, K. Price, Differential Evolution - A Simple and Efficient Heuristic for Global Optimization over Continuous Spaces. {\it Journal of Global Optimization.}, v 11, p 341-359, 1997.

\end{thebibliography}
\end{block}
\end{frame}

\section{Agradecimentos}
\begin{frame}
\begin{block}{\textbf{\textsc{AGRADECIMENTOS}}}
\begin{figure}[h!]
			\centering
			\includegraphics[width=0.8\linewidth]{../imgs/agrade.png}
			\label{figure01}
	\end{figure}
\end{block}
\end{frame}

\end{document}